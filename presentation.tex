\documentclass{beamer}

\usepackage[english]{babel}
\usepackage[utf8]{inputenc}
\usepackage{times}
\usepackage[T1]{fontenc}
\usepackage{listings}
\usepackage{color}

\mode<presentation>
{
  \usetheme{default}
  \title{Compiler Design}
  \subtitle{Functional Programming Languages}
  \author[Ondřej Janošík \and Marek Kidoň]{Ondřej Janošík \and Marek Kidoň}
  \institute{Faculty of Information Technology, Brno University of Technology}
  \date{\today}
}

% Theme tweaks
\usecolortheme{rose}
\useinnertheme[shadow]{rounded}
\usecolortheme{dolphin}

% Custom commands
\newcommand{\vpad}{\vspace{5pt}}
\newcommand{\vPad}{\vspace{10pt}}
\newcommand{\vPAD}{\vspace{15pt}}

\begin{document}

\frame{\maketitle}

\begin{frame} \frametitle{Introduction}

% mention that the list is minimal, it can contain many more items
Imperative languages
  \begin{itemize}
    \item statements
    \item expressions
  \end{itemize}

\vpad
% note this is true in case of purely functional languages only.
Functional languages
  \begin{itemize}
      \item expressions
  \end{itemize}

\vPad
There is no statement-defined control flow nor global state!
% say how is the computation performed then...
\end{frame}

\defverbatim[colored]\passingFunctions{
\begin{lstlisting}[language=haskell,basicstyle=\ttfamily,keywordstyle=\color{blue}]
map :: (a -> b) -> [a] -> [b]
map = ...

power :: Int -> Int
power x = x * x

mapPower :: [Int] -> [Int]
mapPower xs = map power xs
\end{lstlisting}
}

\defverbatim[colored]\curryfication{
\begin{lstlisting}[language=haskell,basicstyle=\ttfamily,keywordstyle=\color{blue}]
map :: (a -> b) -> [a] -> [b]
map = ...

power :: Int -> Int
power x = x * x

mapPower :: [Int] -> [Int]
mapPower = map power
\end{lstlisting}
}

\defverbatim[colored]\lambdaFunction{
\begin{lstlisting}[language=haskell,basicstyle=\ttfamily,keywordstyle=\color{blue}]
map :: (a -> b) -> [a] -> [b]
map = ...

mapPower :: [Int] -> [Int]
mapPower = map (\x -> x * x)
\end{lstlisting}
}

\begin{frame} \frametitle{Functions}
  \framesubtitle{Functions are first-class citizens.}

  \only<1>{They can be passed around as arguments,}
  \only<2>{be result of other functions (using \textit{curryfication}),}
  \only<3>{or anonymous, so called $\lambda$-functions.}

  \begin{overlayarea}{\textwidth}{5cm}
    \begin{block}{}
      \only<1>{\passingFunctions}
      \only<2>{\curryfication}
      % Note: ^ We used eta reduction.
      % Also tell that we are going to describe curryfication few slides later.
      \only<3>{\lambdaFunction}
    \end{block}
  \end{overlayarea}

\end{frame}

\begin{frame} \frametitle{Functions}
  \framesubtitle{Curryfication}

  \begin{block}{Concept}
    % Note: comment on the need of right-associativity and how it works in general.
    \begin{center}
      $(a_{1} \times a_{2} \times \ldots \times a_{n}) \to b$ \\ \vPAD
      vs \\ \vPAD
      $a_{1} \to (a_{2} \to (\ldots \to (a_{n} \to b)))$
    \end{center}
  \end{block}

  \vpad
  % Note: Even a constant value is a function.
  There are only single parameter functions returning function.

\end{frame}


% TODO: Try to find better example
% Note: 'person' is free variable in context of 'name' function.
\defverbatim[colored]\freeVar{
\begin{lstlisting}[language=haskell,basicstyle=\ttfamily,keywordstyle=\color{blue},showstringspaces=false,escapechar=@]
showName :: Person -> Text
showName person = "Name: " <> name
  where
    name = getName @\textcolor{red}{person}
\end{lstlisting}
}

\begin{frame} \frametitle{Functions}
  \framesubtitle{Closures}

  \begin{onlyenv}<1>
    % Note: This depends on context.
    In some functional languages, we might encounter free variables.

    \begin{block}{}
      \freeVar
    \end{block}
  \end{onlyenv}
  \begin{onlyenv}<2>
    We have to use closures to bind free variables.

    \begin{block}{Closure}
      \begin{center}
        $(e, u)$
      \end{center}
      where \\
      \hspace{3pt} e -- argument \\
      \hspace{3pt} u -- environment
      \vpad
    \end{block}
  \end{onlyenv}

\end{frame}

\begin{frame} \frametitle{Functions}
  \framesubtitle{Evaluation strategies}

  Functions can be evaluated in 3 different ways.
  \begin{itemize}
    \item<2-> Applicative order evaluation (call-by-value)
    \item<3-> Normal order evaluation (call-by-name)
    \item<4-> Lazy evaluation (call-by-need)
  \end{itemize}

  % Note: Should this text really be here?
  \vpad
  \begin{block}{}<4>
  Lazy evaluation combines advantages of \textit{Applicative order evaluation}
  and \textit{Normal order evaluation}.
  \end{block}

\end{frame}

\begin{frame}[fragile] \frametitle{Functions}
  \framesubtitle{Putting all together}

  \begin{block}{Possible data structure}
    \begin{verbatim}
union:
  value          -- Evaluated value
  structure:     -- Unevaluated value or
                 -- partially applied function
    function     -- Pointer to function
    arguments[]  -- Array of pointers to
                 -- function arguments
    \end{verbatim}
  \end{block}
\end{frame}

\defverbatim[colored]\fplus{
\begin{lstlisting}[language=haskell,basicstyle=\ttfamily,keywordstyle=\color{blue}]
plus :: Int -> Int -> Int
plus = (+)
\end{lstlisting}
}

\begin{frame} \frametitle{Type system}
  Functional languages are mostly strongly, implicitly typed.

  \vpad
  Each function has type, we can explicitly define the type by providing the
  type signature or let the compiler deduce the type by performing the
  \textit{type inference}.
  \fplus

  There are various type system that support type inference. For example
  the \textit{Hindley-Milner-Damas} type system (derivation used in Haskell).

\end{frame}


\defverbatim[colored]\fpolylist{
\begin{lstlisting}[language=haskell,basicstyle=\ttfamily,keywordstyle=\color{blue}]
length :: [a] -> Int
length [] = 0
length (x:xs) = 1 + length xs
\end{lstlisting}}

\defverbatim[colored]\fpolyfold{
\begin{lstlisting}[language=haskell,basicstyle=\ttfamily,keywordstyle=\color{blue}]
length :: Foldable t => t a -> Int
length  = foldr (const +1) 0
\end{lstlisting}}

\defverbatim[colored]\fpolyrankn{
\begin{lstlisting}[language=haskell,basicstyle=\ttfamily,keywordstyle=\color{blue}]
sumOf2
  :: (forall c. c -> Int)
  -> [b] -> [a] -> Int
sumOf2 f a b = f a + f b
\end{lstlisting}}

\begin{frame} \frametitle{Polymorphism}
  \onslide<1->
  Such type systems often support parametric polymorphism.
  \begin{block}{}
    \fpolylist
  \end{block}

  \onslide<2->
  Polymorphic parameters could be constrained using type-classes.
  \begin{block}{}
    \fpolyfold
  \end{block}
\end{frame}

\begin{frame} \frametitle{Polymorphism}
  How is appropriate implementation selected...

  Some languages introduced higher rank polymorphism.
  \begin{block}{}
    \fpolyrankn
  \end{block}
\end{frame}

\begin{frame} \frametitle{Type inference}
  \framesubtitle{Hindler-Milner-Damas type system}
  Type inference allows us automatically deduce types of \textbf{expressions} in programming languages.
  
  \vPad
  In various functional languages the type inference algorithm is based on the \textit{Hindley-Milner-Damas}
  type system. 

  \vpad
  It is a system providing general type inference algorithm for family of typed $\lambda$-calculus languages.
\end{frame}

\begin{frame} \frametitle{Type inference}
  \framesubtitle{The intuition}
  The code itself places constraints on the types. By using these constraints we can determine the type.
  
  \vpad
  \textbf{Unification} plays significant role since it allows us replace all occourrences of $t1$ with $t2$.
\end{frame}

\defverbatim[colored]\emptyinference{
\begin{lstlisting}[language=haskell,basicstyle=\ttfamily,keywordstyle=\color{blue}]
mkInt :: ? -> ?
mkInt a = if a then 1 else 0
\end{lstlisting}}

\defverbatim[colored]\booledinference{
\begin{lstlisting}[language=haskell,basicstyle=\ttfamily,keywordstyle=\color{blue}]
mkInt :: Bool -> ?
mkInt a = if a then 1 else 0
\end{lstlisting}}

\defverbatim[colored]\intedinference{
\begin{lstlisting}[language=haskell,basicstyle=\ttfamily,keywordstyle=\color{blue}]
mkInt :: Bool -> Int
mkInt a = if a then 1 else 0
\end{lstlisting}}

\defverbatim[colored]\classinference{
\begin{lstlisting}[language=haskell,basicstyle=\ttfamily,keywordstyle=\color{blue}]
mkInt :: Num a => Bool -> a
mkInt a = if a then 1 else 0
\end{lstlisting}}

\begin{frame} \frametitle{Type inference}
  \framesubtitle{Example}

  \only<1>{\textit{mkInt} is a function of 1 argument consisting of the $if$ expression only.}
  \only<2>{First argument of the $if$ expression is bool.}
  \only<3>{Both the second and third argument are of the same type $Int$}
  \only<4>{Real languages use variations of this algorithm, they would infer 
    something more general.}
  
  \begin{overlayarea}{\textwidth}{5cm}
    \begin{block}{}
      \only<1>{\emptyinference}
      \only<2>{\booledinference}
      \only<3>{\intedinference}
      \only<4>{\classinference}
    \end{block}
  \end{overlayarea}

\end{frame}

\end{document}
