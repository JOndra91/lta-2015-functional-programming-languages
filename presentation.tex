\documentclass{beamer}

\usepackage[english]{babel}
\usepackage[utf8]{inputenc}
\usepackage{times}
\usepackage[T1]{fontenc}

\mode<presentation>
{
  \usetheme{default}
  \title{Compiler Design}
  \subtitle{Functional Programming Languages}
  \author[Ondřej Janošík \and Marek Kidoň]{Ondřej Janošík \and Marek Kidoň}
  \institute{Faculty of Information Technology, Brno University of Technology}
  \date{\today}
}

\begin{document}

\frame{\maketitle}

\begin{frame} \frametitle{Introduction}

% mention that the list is minimal, it can contain many more items
Imperative languages
  \begin{itemize}
    \item statements
    \item expressions
  \end{itemize}

\vspace{5pt}
% note this is true in case of purely functional languages only.
Functional languages
  \begin{itemize}
      \item expressions
  \end{itemize}

\vspace{10pt}
There is no statement-defined control flow nor global state!
% say how is the computation performed then...
\end{frame}

\begin{frame} \frametitle{Basic constructs [1/2]}
  Functions are first-class citizens.
  \begin{itemize}
      \item They can be passed around as arguments,
      \item partially applied using \textit{curryfication},
      \item or unnamed, so called \textit{anonymous} or $\lambda$-functions.
  \end{itemize}

  \vspace{10pt}
  Curryfication concept
  % typewriting TODO
  % animations are welcome
  % comment on the need of right-associativity and how it works in general.
  $(a_{1} \times a_{2} \times \ldots \times a_{n}) \to r$
  vs
  $a_{1} \to a_{2} \to \ldots \to a_{n} \to r$

  % for example.
  \vspace{5pt}
  % even a constant value is a function.
  In Haskell there are only single parameter returning functions.
\end{frame}

\begin{frame} \frametitle{2/2}

\end{frame}

\begin{frame} \frametitle{Polymorphism}
  

\end{frame}


\end{document}
